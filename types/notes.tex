\documentclass[a4paper,11pt]{article}

\usepackage{amsmath}
\usepackage{amssymb}
\usepackage[utf8]{inputenc}
\usepackage[margin=2cm]{geometry}
\usepackage{prftree}
\usepackage{changepage}

\title{\vspace{-1cm}Types\vspace{-1cm}}
\author{}
\date{}

\setlength{\parindent}{0cm}
\setlength{\parskip}{2mm}

% Make ~ look more normal
\let\oldsim\sim 
\renewcommand{\sim}{{\oldsim}}

\begin{document}
\maketitle
\section{Proof}
{
    \subsubsection*{Structural Induction}
    {

        Proof based on the rules, so `structural induction on \(\vdash\)' not `on \(\Gamma\vdash e : \tau\)'.

        For each rule, the `case' is the rule. We assume that both the judgements on the top hold (as they're subterms)
        and that the hypothesis holds \textbf{on those terms}.
    
    }
}
\section{Terminology}
{
    \begin{description}
    \item[Congruence Rule]\hfill\\A rule for \(\leadsto\) that reduces a subterm, like \(e_1\;e_2 \leadsto e_1'\;e_2\).
    \item[Reduction Rule]\hfill\\A rule for \(\leadsto\) that transform the term, like \((\lambda x:X.\;e)\;v \leadsto
    e[v/x]\).
    \end{description}
}
\section{Theorems}
{
    \begin{description}
    \item[Weakening]
    {
        \hfill

        If \(\Gamma,\Gamma' \vdash e : \tau\) then \(\Gamma, x : \tau'', \Gamma' \vdash e : \tau\).

        If a term typechecks under a context, then it typechecks under a larger context.
    }
    \item[Exchange]
    {
        \hfill

        If \(\Gamma, x_1 : \tau_1, x_2 : \tau_2, \Gamma' \vdash e : \tau\) then \(\Gamma, x_2 : \tau_2, x_1 : \tau_1,
        \Gamma' \vdash e : \tau\).

        If a term typechecks in a context, then it typechecks in a reordered context.

    }
    \item[Substitution]
    {
        \hfill
        
        If \(\Gamma \vdash e : \tau\) and \(\Gamma, x : \tau \vdash e' : \tau'\) then \(\Gamma \vdash [e/x]e' : \tau'\).

        Substituting a type-correct term for a variable preserves type-correctness.
    }
    \item[Progress]
    {
        \hfill

        If \(\cdot \vdash e : \tau\) then either \(e\) is a value or \(\exists e'.\; e \leadsto e'\).

        Type-correct terms are never stuck. Note that it's \(\cdot\) not any \(\Gamma\).
    }
    \item[Preservation]
    {
        \hfill

        If \(\cdot \vdash e : \tau\) and \(\exists e'.\; e \leadsto e'\) then \(\cdot \vdash e' : \tau\).

        Evaluation preserves types (evaluation may be non-terminating). Note that it's \(\cdot\) not any \(\Gamma\).

        Proof by structural induction on \(e \leadsto e'\).
    }
    \item[Determinacy]
    {
        \hfill

        If \(e \leadsto e'\) and \(e \leadsto e''\) then \(e' = e''\).
        
        Proof by structural induction on \(e \leadsto e'\).
    }
    \item[Type Safety] Progresss + Preservation
    \item[Consistency]
    {
        \hfill

        There are no proofs of \(\bot\), or there are no terms \(e\) such that \(\cdot\vdash e : 0\).

        Possible in all good logics, but only possible in \textbf{total} languages.
    }
    \item[Curry-Howard Correspondence]
    {
        \hfill

        Programs are proofs, types are formulae. Not an isomorphism, consider \(X \Longrightarrow (Y \Longrightarrow X \wedge Y)\).
    }
    \item[Type Weakening] If \(\Theta,\Theta'\vdash A\text{ type}\) then \(\Theta,\beta,\Theta'\vdash A\text{ type}\).
    \item[Type Exchange] If \(\Theta,\beta,\gamma,\Theta'\vdash A\text{ type}\) then \(\Theta,\gamma,\beta,\Theta'\vdash
    A\text{ type}\).
    \item[Type Substitution] If \(\Theta\vdash A\text{ type}\) and \(\Theta,\alpha\vdash B\text{ type}\) then
    \(\Theta\vdash B[A/\alpha]\text{ type}\).
    \item[Context Weakening] If \(\Theta,\Theta'\vdash \Gamma\text{ ctx}\) then \(\Theta,\alpha,\Theta'\vdash \Gamma\text{ ctx}\).
    \item[Context Exchange] If \(\Theta,\beta,\gamma,\Theta'\vdash \Gamma\text{ ctx}\) then
    \(\Theta,\gamma,\beta,\Theta'\vdash \Gamma\text{ ctx}\).
    \item[Context Substitution] If \(\Theta\vdash A\text{ type}\) and \(\Theta,\alpha\vdash \Gamma\text{ ctx}\) then
    \(\Theta\vdash \Gamma[A/\alpha]\text{ ctx}\).
    \item[Regularity]
    {
        \hfill
        
        If \(\Theta\vdash\Gamma\text{ ctx}\) and \(\Theta;\Gamma\vdash e:A\) then \(\Theta\vdash A\text{ type}\).

        Typechecking only succeeds for well-formed types.

        Proof by structural induction on \(\Theta;\Gamma\vdash e:A\).
    
    }
    \end{description}
}
\section{Recursive Functions}
{
    \begin{gather*}
    v ::= ... \mid \text{fun}_{X \rightarrow Y}\;f\;x.\;e
    \\
    \\
    \prftree[r]{(Fix)}
    {\Gamma,f:X \rightarrow Y,x:X \vdash e : Y}
    {\Gamma\vdash \text{fun}_{X \rightarrow Y}\;f\;x.\;e : X \rightarrow Y}
    \hspace{1cm}
    \prftree[r]{}
    {(\text{fun}_{X \rightarrow Y}\;f\;x.\;e) v \leadsto e[v/x, (\text{fun}_{X \rightarrow Y}\;f\;x.\;e)/f]}
    \end{gather*}

    Natural for defining functions: \(\text{fun}_{\texttt{int}\rightarrow\texttt{int}}\;fact\;n.\;\text{if }n = 0
    \text{ then }1\text{ else }n \times fact (n-1)\). Also allows for infinite recursion, so no longer total/consistent.

}
\section{The Halt Relation}
{

    \(e\) halts iff \(\exists v.\; e \leadsto^* v\).

    Define a type-indexed family of sets of terms: 
    \begin{itemize}
    \item \(\text{Halt}_0 = \emptyset\): there are no terms of type \(0\) that halt.
    \item \(e \in \text{Halt}_1\) only when \(e\) halts and \(\cdot \vdash e : 1\).
    \item \(e \in \text{Halt}_{X \rightarrow Y}\) only when \(e\) halts, \(\cdot \vdash e : X \rightarrow Y\) , and
    \(\forall e'.\; e' \in \text{Halt}_X \Longrightarrow (e\;e') \in \text{Halt}_Y\).
    \end{itemize}

    All terms in \(\text{Halt}_1\) halt, all terms in \(\text{Halt}_{X \rightarrow Y}\) are functions which preserve the
    property of halting, ...

    \subsection*{Halting}
    {
        If \(e\in\text{Halt}_X\) then \(e\) halts (\(\exists v.\;e \leadsto^* v\)).

    }
    \subsection*{Closure Lemma}
    {
        If \(e \leadsto e'\) then \(e \in \text{Halt}_X \Longleftrightarrow e' \in \text{Halt}_X\).
        
        Proof by induction on the `type indices' to the sets, ie.\ by structural induction where we need to prove the
        lemma for the types \(X = 1, Y \rightarrow Z, ...\), assuming for the IH that \(\text{Halt}_t\) holds for all
        subtypes \(t\) of \(X\).

    }
    \subsection*{The Fundamental Lemma}
    {
        
        If \(\forall n \geq 0\), \(x_1:X_1,...,x_n:X_n \vdash e : Z\) and \(\forall i\in[1..n]\), \(\cdot\vdash
        v_i:X_i\) and \(v_i \in \text{Halt}_{X_i}\), then \([v_1/x_1,...,v_n/x_n]e \in \text{Halt}_Z\).

        Proof by structural induction on \(x_1:X_1,...,x_n:X_n \vdash e : Z\) (go through the rules).

        This is powerful as it says that \textit{all well-typed terms terminate}.

    }
    \subsection*{Consistency}
    {
        There are no terms \(e\) such that \(\Gamma\vdash e : 0\). Proof by contradiction using the fundamental lemma:

        Assume \(\Gamma\vdash e : 0\), then by the fundamental lemma (with \(n=0\)) \(e \in \text{Halt}_0\). But by the
        definition of \(\text{Halt}_0\), \(e\notin\text{Halt}_0\), so contradiction.

    }
}
\section{Halt for System F}
{

    Similar to Halt for STLC, but uses a semantic interpretation to define the sets. Probably too complex to be expected
    to remember, rules would be given in exam.

}
\section{Datatypes}
{
    \subsection{STLC}
    {
        \begin{tabular}{c | l}
        Builtin & Encoding \\
        \hline
        Bool & \(1+1\) \\
        True & L \(\langle\rangle\) \\
        False & R \(\langle\rangle\) \\
        if \(e\) then \(e'\) else \(e''\) & \(\text{case}(e,\text{L \_}\rightarrow e',\text{R \_}\rightarrow e'')\) \\
        \\
        Char & Bool\(^7\) \\
        `A' & (True,False,False,False,False,False,True) \\
        ... & ...
        \end{tabular}
    }
    \subsection{PLC}
    {

        \begin{tabular}{c | l}
        Builtin & Encoding \\
        \hline
        Bool & \(\forall\alpha.\;\alpha\rightarrow\alpha\rightarrow\alpha\) \\
        True & \(\Lambda\alpha.\;\lambda x:\alpha.\;\lambda y:\alpha.\;x\) \\
        False & \(\Lambda\alpha.\;\lambda x:\alpha.\;\lambda y:\alpha.\;y\) \\
        if \(e\) then \(e'\) else \(e''\) \(: X\) & \(e\;X\;e'\;e''\) \\
        \\
        \(X \times Y\) & \(\forall\alpha.\;(X \rightarrow Y \rightarrow \alpha) \rightarrow \alpha\) \\
        \(\langle e,e' \rangle\) & \(\Lambda\alpha. \lambda k:X \rightarrow Y \rightarrow \alpha.\; k\;e\;e'\) \\
        fst \(e\) & \(e\;X\;(\lambda x:X.\;\lambda y:Y.\;x)\) \\
        snd \(e\) & \(e\;X\;(\lambda x:X.\;\lambda y:Y.\;y)\) \\
        \\
        \(X+Y\) & \(\forall\alpha.\;(X\rightarrow\alpha)\rightarrow(Y\rightarrow\alpha)\rightarrow\alpha\) \\
        L \(e\) & \(\Lambda\alpha.\;\lambda f:(X\rightarrow\alpha).\;\lambda g:(Y\rightarrow\alpha).\;f\;e\) \\
        R \(e\) & \(\Lambda\alpha.\;\lambda f:(X\rightarrow\alpha).\;\lambda g:(Y\rightarrow\alpha).\;g\;e\) \\
        case\((e,\text{L }x\rightarrow e_1,\text{R }y\rightarrow e_2) : Z\) & \(e\;Z\;(\lambda x:(X \rightarrow
        Z).\;e_1)\;(\lambda y:(Y \rightarrow Z).\;e_2)\) \\
        \\
        \(\mathbb{N}\) & \(\forall\alpha.\;\alpha\rightarrow(\alpha\rightarrow\alpha)\rightarrow\alpha\) \\
        z & \(\Lambda\alpha.\;\lambda z:\alpha.\;\lambda s:(\alpha\rightarrow\alpha).\;z\) \\
        s\((e)\) & \(\Lambda\alpha.\;\lambda z:\alpha.\;\lambda s:(\alpha\rightarrow\alpha).\;s\;(e\;\alpha\;z\;s)\) \\
        iter\((e,\text{z}\rightarrow e_z,\text{s}(x)\rightarrow e_s) : X\) & \(e\;X\;e_z\;(\lambda x:X.\;e_s)\) \\
        ... & ...
        \end{tabular}

        The function(s) `attached' to the encoded datatypes are selection operations: for Bool it's switching, for pairs
        it's selecting, for sums it's conditional.
        \begin{align*}
        \text{if True then }e'\text{ else } e'' : A &= \text{True}\;A\;e'\;e'' \\
                                    &= (\Lambda\alpha.\;\lambda x:\alpha.\;\lambda y:\alpha.\;x)\;A\;e'\;e'' \\
                                    &= (\lambda x:A.\;\lambda y:A.\;x)\;e'\;e'' \\
                                    &= (\lambda y:A.\;e')\;e'' \\
                                    &= e' \\
        \end{align*}
    }
}
\section{Existentials}
{
    \begin{alignat*}{4}
    &\text{Types } &&A &&::= ... \mid \exists\alpha.\;A && \\
    &\text{Terms } &&e &&::= ... \mid \text{pack}_{\alpha.A}(C,e) \mid \text{let pack}(\alpha,x) = e\text{ in }e' && \\
    &\text{Values } &&v &&::= \text{pack}_{\alpha.A}(C,v) &&
    \end{alignat*}

    \begin{gather*}
    \prftree[r]{(\(\exists_\text{I}\))}
    {\Theta,\alpha\vdash A\text{ type}}
    {\Theta\vdash C\text{ type}}
    {\Theta;\Gamma\vdash e:A[C/\alpha]}
    {\Theta;\Gamma\vdash \text{pack}_{\alpha,A}(C,e) : \exists\alpha.\;A}
    \\
    \\
    \prftree[r]{(\(\exists_\text{I}\))}
    {\Theta;\Gamma\vdash e:\exists\alpha.\;A}
    {\Theta,\alpha;\Gamma,x:A \vdash e' : D}
    {\Theta\vdash D\text{ type}}
    {\Theta;\Gamma\vdash \text{let pack}(\alpha,x) = e\text{ in }e':D}
    \\
    \\
    \prftree[r]{}
    {\text{let pack}(\alpha,x) = \text{pack}_{\alpha,A}(C,v) \text{ in } e \leadsto e[C/\alpha,v/x]}
    \end{gather*}

    \(\alpha\) is a placeholder for the concrete type, \(C\) is the \textit{concrete} type, and \(A\) is the
    \textit{abstract} interface.

    \begin{itemize}
    \item
    {

        \(\text{pack}_{\alpha,A}(C,e)\) packs a implementation \(e\) with concrete type \(C\) into a provider for the
        \(\alpha,A\) interface.

    }
    \item
    {

        \(\text{let pack}(\alpha, x) = \text{pack}_{\alpha,A}(C,v)\text{ in }e\) unpacks a provider for the \(\alpha,A\)
        interface into the concrete type and implementation \(\alpha\) and \(x\).

    }
    \end{itemize}

    \subsection*{Encoding}
    {

        \begin{tabular}{l | l}
        Original & Encoding \\
        \hline
        \(\exists\alpha.\;A\) & \(\forall\beta.\;(\forall\alpha.\;A\rightarrow\beta)\rightarrow\beta\) \\
        \(\text{pack}_{\alpha,A}(C,e)\) & \(\Lambda\beta.\;\lambda k:(\forall\alpha.\;A\rightarrow\beta).\;k\;C\;e\) \\
        \(\text{let pack}(\alpha,x) = e\text{ in }e' : D\) & \(e\;D\;(\Lambda\alpha.\;\lambda x:A.\;e')\) \\
        \end{tabular}
    }
}
\section{State}
{

    \begin{alignat*}{4}
    &\text{Types } &&X &&::= ... \mid \text{ref }X && \\
    &\text{Terms } &&e &&::= ... \mid \text{new }e \mid\;!e \mid e := e' \mid l && \\
    &\text{Values } &&v &&::= ... \mid l && \\
    &\text{Stores } &&\sigma &&::= \cdot \mid \sigma,l:v && \\
    &\text{Store Typings } &&\Sigma &&::= \cdot \mid \Sigma,l:X &&
    \end{alignat*}
    
    \begin{gather*}
    \prftree[r]{}
    {\Sigma;\Gamma\vdash e:X}
    {\Sigma;\Gamma\vdash \text{new }e:\text{ref }X}
    \qquad
    \prftree[r]{}
    {l:X \in \Sigma}
    {\Sigma;\Gamma\vdash l : \text{ref }X}
    \\
    \\
    \prftree[r]{}
    {\Sigma;\Gamma\vdash e:\text{ref }X}
    {\Sigma;\Gamma\vdash\;!e:X}
    \qquad
    \prftree[r]{}
    {\Sigma;\Gamma\vdash e:\text{ref }X}
    {\Sigma;\Gamma\vdash e':X}
    {\Sigma;\Gamma\vdash\;e:=e':1}
    \end{gather*}

    Operational semantics are for \(\langle \sigma,e \rangle\).

    There's a store well-typing rule but the notes mess it up: essentially for each store element, compare the type
    inferred from the value to the type stored in the store context \(\Sigma\).

    \subsection*{Preservation}
    {
        The normal preservation property no longer holds:

        \(e = \langle\cdot;\text{new }\langle\rangle\rangle : \langle\cdot;\text{ref }1\rangle\), but \(e\leadsto
        \langle l:\langle\rangle;l\rangle : \langle l:1;\text{ref }1\rangle\), where the stores don't match.

        \subsubsection*{Store Extension}
        {
            Define \(\Sigma \leq \Sigma'\) to mean \(\exists\Sigma''.\; \Sigma'=\Sigma,\Sigma''\).
        }
        \subsubsection*{Store Monotonicity}
        {
            If \(\Sigma\leq\Sigma'\) then:
            \begin{enumerate}
            \item If \(\Sigma;\Gamma\vdash e : X\) then \(\Sigma';\Gamma\vdash e : X\).
            \item If \(\Sigma\vdash \sigma_0 : \Sigma_0\) then \(\Sigma'\vdash \sigma_0 : \Sigma_0\).
            \end{enumerate}

            Allocating new references never breaks term typability.
        }
        \subsubsection*{Preservation Repaired}
        {

            If \(\langle\sigma;e\rangle : \langle\Sigma;X\rangle\) and
            \(\langle\sigma;e\rangle\leadsto\langle\sigma';e'\rangle\), then there exists a \(\Sigma'\geq\Sigma\) such
            that \(\langle\sigma';e'\rangle : \langle\Sigma';X\rangle\).

        }
    }
    \subsection*{Recursion}
    {

        With state we can perform recursion even in the STLC, by storing and updating a function in a reference.

    }
}
\section{Monads}
{
    Separate judgements between pure and impure computations, to prevent pure code being able to perform impure.

    \begin{alignat*}{4}
    &\text{Types } &&X &&::= ... \mid \text{ref }X \mid T\;X && \\
    &\text{Pure Terms } &&e &&::= ... \mid l \mid \{t\} && \\
    &\text{Impure Terms } &&t &&::= \text{new }e \mid \;!e \mid e:=e' \mid \text{let }x=e;\;t \mid \text{return }e && \\
    &\text{Values } &&t &&::= ... \mid l \mid \{t\} &&
    \end{alignat*}

    \begin{gather*}
    \prftree[r]{}
    {\Sigma;\Gamma\vdash t \div X}
    {\Sigma;\Gamma\vdash \{t\} : T\;X}
    \\
    \\
    \prftree[r]{}
    {\Sigma;\Gamma\vdash e : \text{ref }X}
    {\Sigma;\Gamma\vdash \;!e \div X}
    \qquad
    \prftree[r]{}
    {\Sigma;\Gamma\vdash e : X}
    {\Sigma;\Gamma\vdash \text{return }e \div X}
    \qquad
    \prftree[r]{}
    {\Sigma;\Gamma\vdash e : T\;X}
    {\Sigma;\Gamma,x:X \vdash t \div X}
    {\Sigma;\Gamma\vdash\text{let }x=e;\;t\div Z}
    \qquad
    ...
    \end{gather*}

    \(\text{let }x=e;\;t\div Z\) evaluates a monadic effect within other impure code. \textbf{No escape from the monad},
    though.

    All the impure terms are typed by \(\div\): no introduction of the \(T\) monad yet. The \(\{t\}\) term in the pure
    terms converts a judgement of \(\div\) into a monadic type so it can be passed around pure code without tainting it.

    Proofs are by mutual induction, as there are distinct properties for pure and impure terms.
}
\section{Classical Logic}
{
    \subsection*{Logic}
    {

        Step away from the intuitionistic view of refutations being deriving (\(\neg P
        \stackrel{\text{def}}{=}P\Longrightarrow\bot\)) by making them first-class.

        \begin{alignat*}{4}
        &\text{Propositions } &&A && ::= \top \mid \bot \mid \neg A \mid A \wedge B \mid A \vee B && \\
        &\text{True Contexts } &&\Gamma && ::= \cdot \mid \Gamma,A && \\
        &\text{False Contexts } &&\Delta && ::= \cdot \mid \Delta,A &&
        \end{alignat*}

        A proof is \(\Gamma;\Delta\vdash A\text{ true}\): if things in \(\Gamma\) are true and things in \(\Delta\) are
        false, then \(A\) is true. Similarly for \(\Gamma;\Delta\vdash A\text{ false}\).

        A refutation is \(\Gamma;\Delta\vdash \text{contr}\), when \(\Gamma\) and \(\Delta\) contradict one another.

        Note that there's no implication: can derive it as \(A \Longrightarrow B\stackrel{\text{def}}{=}\neg A \vee B\).

        \begin{minipage}[t]{0.49\textwidth}
        \centering
        Proofs
        \begin{gather*}
        \prftree[r]{}
        {A\in\Gamma}
        {\Gamma;\Delta \vdash A \text{ true}}
        \qquad
        \prftree[r]{}
        {\Gamma;\Delta \vdash \top \text{ true}}
        \\
        \\
        \prftree[r]{}
        {\Gamma;\Delta\vdash A\text{ true}}
        {\Gamma;\Delta\vdash B\text{ true}}
        {\Gamma;\Delta\vdash A \wedge B\text{ true}}
        \\
        \\
        \prftree[r]{}
        {\Gamma;\Delta\vdash A\text{ true}}
        {\Gamma;\Delta\vdash A \vee B\text{ true}}
        \qquad
        \prftree[r]{}
        {\Gamma;\Delta\vdash B\text{ true}}
        {\Gamma;\Delta\vdash A \vee B\text{ true}}
        \\
        \\
        \prftree[r]{}
        {\Gamma;\Delta\vdash A\text{ false}}
        {\Gamma;\Delta\vdash \neg A\text{ true}}
        \end{gather*}
        \end{minipage}
        \begin{minipage}[t]{0.49\textwidth}
        \centering
        Refutations
        \begin{gather*}
        \prftree[r]{}
        {A\in\Delta}
        {\Gamma;\Delta \vdash A \text{ false}}
        \qquad
        \prftree[r]{}
        {\Gamma;\Delta \vdash \bot \text{ false}}
        \\
        \\
        \prftree[r]{}
        {\Gamma;\Delta\vdash A\text{ false}}
        {\Gamma;\Delta\vdash B\text{ false}}
        {\Gamma;\Delta\vdash A \vee B\text{ false}}
        \\
        \\
        \prftree[r]{}
        {\Gamma;\Delta\vdash A\text{ false}}
        {\Gamma;\Delta\vdash A \wedge B\text{ false}}
        \qquad
        \prftree[r]{}
        {\Gamma;\Delta\vdash B\text{ false}}
        {\Gamma;\Delta\vdash A \wedge B\text{ false}}
        \\
        \\
        \prftree[r]{}
        {\Gamma;\Delta\vdash A\text{ true}}
        {\Gamma;\Delta\vdash \neg A\text{ false}}
        \end{gather*}
        \end{minipage}

        These allow proving \textbf{most} of classical logic, but not all: we can't prove \(\neg\neg A;\cdot \vdash A\),
        or \(A \wedge B \Longrightarrow A\).

        Introduce contradiction rules:

        \begin{gather*}
        \prftree[r]{}
        {\Gamma;\Delta,A\vdash\text{ contr}}
        {\Gamma;\Delta\vdash A\text{ true}}
        \qquad
        \prftree[r]{}
        {\Gamma,A;\Delta\vdash\text{ contr}}
        {\Gamma;\Delta\vdash A\text{ false}}
        \\
        \\
        \prftree[r]{}
        {\Gamma;\Delta\vdash A\text{ true}}
        {\Gamma;\Delta\vdash A\text{ false}}
        {\Gamma;\Delta\vdash \text{ contr}}
        \end{gather*}

    }
    \subsection*{Language}
    {

        STLC + continuations. Essentially mirrors the logic:

        \begin{alignat*}{4}
        &\text{Propositions } &&X &&::= \top \mid \bot \mid X \wedge Y \mid X \vee Y && \\
        &\text{True Contexts } &&\Gamma &&::= \cdot \mid \Gamma,x:A && \\
        &\text{False Contexts } &&\Delta &&::= \cdot \mid \Delta,u:A && \\
        \\
        &\text{Expressions } &&e &&::= \langle\rangle \mid \langle e,e' \rangle \mid \text{L }e \mid \text{R }e \mid
        \text{not}(k) \mid \mu u : A.\; c && \\
        &\text{Continuations } &&k &&::= [] \mid [e,e'] \mid \text{fst }e \mid \text{snd }e \mid \text{not}(e) \mid \mu
        x : A.\; c && \\
        &\text{Contradictions } &&c &&::= \langle e \mid_A k \rangle &&
        \end{alignat*}

        \begin{minipage}[t]{0.49\textwidth}
        \footnotesize
        \centering
        Proof Terms
        \begin{gather*}
        \prftree[r]{}
        {x:A\in\Gamma}
        {\Gamma;\Delta\vdash x:A\text{ true}}
        \qquad
        \prftree[r]{}
        {\Gamma;\Delta\vdash \langle\rangle:\top\text{ true}}
        \\
        \\
        \prftree[r]{}
        {\Gamma;\Delta\vdash e :A\text{ true}}
        {\Gamma;\Delta\vdash e' :A\text{ true}}
        {\Gamma;\Delta\vdash \langle e,e' \rangle :A \wedge B\text{ true}}
        \\
        \\
        \prftree[r]{}
        {\Gamma;\Delta\vdash e:A\text{ true}}
        {\Gamma;\Delta\vdash \text{L }e:A \vee B\text{ true}}
        \qquad
        \prftree[r]{}
        {\Gamma;\Delta\vdash e:B\text{ true}}
        {\Gamma;\Delta\vdash \text{R }e:A \vee B\text{ true}}
        \\
        \\
        \prftree[r]{}
        {\Gamma;\Delta\vdash k:A\text{ false}}
        {\Gamma;\Delta\vdash \text{not}(k):A\text{ true}}
        \\
        \end{gather*}
        \end{minipage}
        \begin{minipage}[t]{0.49\textwidth}
        \footnotesize
        \centering
        Refutation Terms
        \begin{gather*}
        \prftree[r]{}
        {x:A\in\Delta}
        {\Gamma;\Delta\vdash x:A\text{ false}}
        \qquad
        \prftree[r]{}
        {\Gamma;\Delta\vdash []:\bot\text{ false}}
        \\
        \\
        \prftree[r]{}
        {\Gamma;\Delta\vdash k :A\text{ false}}
        {\Gamma;\Delta\vdash k' :A\text{ false}}
        {\Gamma;\Delta\vdash [k,k'] :A \vee B \text{ false}}
        \\
        \\
        \prftree[r]{}
        {\Gamma;\Delta\vdash k:A\text{ false}}
        {\Gamma;\Delta\vdash \text{fst }k:A \wedge B\text{ false}}
        \qquad
        \prftree[r]{}
        {\Gamma;\Delta\vdash k:B\text{ false}}
        {\Gamma;\Delta\vdash \text{snd }k:A \wedge B\text{ false}}
        \\
        \\
        \prftree[r]{}
        {\Gamma;\Delta\vdash e:A\text{ true}}
        {\Gamma;\Delta\vdash \text{not}(e):A\text{ false}}
        \\
        \end{gather*}
        \end{minipage}

        Contradictions are also similar. Note that \(\Gamma;\Delta\vdash c\text{ contr}\) is a completely different
        judgement than \(\Gamma;\Delta\vdash c\text{ true}\).

        A contradiction \(\langle e \mid_A k \rangle\) is a term containing two terms which prove the truth and
        falsehood of a type \(A\).

        \begin{gather*}
        \prftree[r]{}
        {\Gamma;\Delta,u:A\vdash c\text{ contr}}
        {\Gamma;\Delta\vdash (\mu u:A.\;c):A\text{ true}}
        \qquad
        \prftree[r]{}
        {\Gamma,x:A;\Delta\vdash c\text{ contr}}
        {\Gamma;\Delta\vdash (\mu x:A.\;c):A\text{ false}}
        \\
        \\
        \prftree[r]{}
        {\Gamma;\Delta\vdash e:A \text{ true}}
        {\Gamma;\Delta\vdash k:A \text{ false}}
        {\Gamma;\Delta\vdash \langle e \mid_A k \rangle \text{ contr}}
        \\
        \end{gather*}

        Operational semantics for contradictions \(\langle e \mid_A k \rangle\):

        \begin{gather*}
        \langle\langle e_1,e_2 \rangle \mid_{A \wedge B} \text{fst }k\rangle \leadsto \langle e_1 \mid_A k \rangle
        \qquad
        \langle \text{R }e \rangle \mid_{A \wedge B} [k_1,k_2] \rangle \leadsto \langle e \mid_A k_2 \rangle
        \\
        \\
        \langle\langle e_1,e_2 \rangle \mid_{A \wedge B} \text{snd }k\rangle \leadsto \langle e_2 \mid_A k \rangle
        \qquad
        \langle \text{L }e \rangle \mid_{A \wedge B} [k_1,k_2] \rangle \leadsto \langle e \mid_A k_1 \rangle
        \\
        \\
        \langle \text{not}(k) \mid_{\neg A} \text{not}(e) \rangle \leadsto \langle e \mid_A k \rangle
        \\
        \\
        \langle \mu u:A.\;c \mid_A k \rangle \leadsto c[k/u]
        \qquad
        \langle e \mid_A \mu x:A.\;c \rangle \leadsto c[e/x]
        \end{gather*}

        The term used in contradictions is normally something relatively complex that might already exist as an
        assumption. Before starting the contradiction, \textbf{break down the RHS} as much as possible: the smaller the
        thing you end up assuming for contradiction is, the more generally useful it is. For example:
        \begin{gather*}
        \prftree[r]{}
        {
            \prftree[r]{}
            {...\textit{hard}...}
            {\neg(\neg A \wedge \neg B);A \vee B \vdash \text{contr}}
        }
        {\neg(\neg A \wedge \neg B);\cdot \vdash A \vee B\text{ true}}
        \qquad
        \prftree[r]{}
        {
            \prftree[r]{}
            {
                \prftree[r]{}
                {...\textit{way easier}...}
                {\neg(\neg A \wedge \neg B);A \vee B \vdash \text{contr}}
            }
            {\neg(\neg A \wedge \neg B);\cdot \vdash A\text{ true}}
        }
        {\neg(\neg A \wedge \neg B);\cdot \vdash A \vee B\text{ true}}
        \end{gather*}

        Don't forget to use the \textbf{normal rules} like \(\vee\) and \(\wedge\).

        To construct terms, normally easiest to prove the theorem using the logic, then mirror the tree in the syntax.
        For example:
        \begin{equation*}
        \prftree[r]{}
        {
            \prftree[r]{}
            {
                \prfaxiom{A \wedge B;A\vdash A\wedge B\text{ true}}
            }
            {
                \prftree[r]{}
                {
                    \prfaxiom{A \wedge B;A\vdash A\text{ false}}
                }
                {A \wedge B;A\vdash A\wedge B\text{ false}}
            }
            {A \wedge B;A\vdash \text{ contr}}
        }
        { A \wedge B;\cdot \vdash A\text{ true}}
        \end{equation*}

        A term proving this is then \(\mu u:A.\;\langle x \mid_{A \wedge B} \text{L }u \rangle\), where \(x\) is the
        assumed proof of \(A \wedge B\). The typing derivation is:
        \begin{equation*}
        \prftree[r]{}
        {
            \prftree[r]{}
            {
                \prfaxiom{x:A \wedge B;u:A\vdash x: A\wedge B\text{ true}}
            }
            {
                \prftree[r]{}
                {
                    \prfaxiom{x:A \wedge B;u:A\vdash u:A\text{ false}}
                }
                {x:A \wedge B;u:A\vdash \text{L }u : A\wedge B\text{ false}}
            }
            {x:A \wedge B;u:A\vdash \langle x \mid_{A \wedge B} \text{L }u \rangle \text{ contr}}
        }
        {x:A \wedge B;\cdot \vdash (\mu u:A.\;\langle x \mid_{A \wedge B} \text{L }u \rangle) : A\text{ true}}
        \end{equation*}
    }
}
\section{Embedding Classical Logic into Intuitionistic Logic}
{
    
    Translating from the continuation language into STLC.

    Pick an arbitrary proposition \(p\) to act as false. Define \(\sim X \stackrel{\text{def}}{=}X \rightarrow P\).

    Can \textbf{translate} types from classical to intuitionistic using:

    \begin{align*}
    (\neg A)^\circ &= \sim A^\circ \\
    \top^\circ     &= 1 \\
    (A \wedge B)^\circ &= A^\circ \times B^\circ \\
    \bot^\circ     &= p \\
    (A \vee B)^\circ &= \sim\sim(A^\circ+B^\circ)
    \end{align*}

    Note that \(\sim\sim\sim X\) isn't \(X \rightarrow X \rightarrow X \rightarrow p\), it's \(((X \rightarrow p)
    \rightarrow p) \rightarrow p\).

    The type \(\neg\neg X \rightarrow X\) can't be translated into a constructive logic, but \(\sim\sim\sim X
    \rightarrow \sim X\) can be (ignoring types for clarity):
    \[ \text{tne} = \lambda f.\; \lambda a.\; f (\lambda g.\; g\;a) \]

    We can construct a function \(\text{dne}_A : \sim\sim A^\circ \rightarrow A^\circ\) for every type \(A\) by using
    tne and structural induction (as we're talking about propositional logic, we have to define it for all the infinite
    types rather than quantifying):

    \begin{alignat*}{2}
    & \text{dne}_\top &&= \lambda q : 1.\; \langle\rangle \\
    & \text{dne}_\bot &&= \lambda q : p.\; q\;(\lambda x.\;x) \\
    & \text{dne}_{A \wedge B} &&= \lambda q : (A^\circ \times B^\circ).\; \langle \text{dne}_A(\text{fst }p), \text{dne}_B(\text{snd }q) \rangle \\
    & \text{dne}_{A \vee B} &&= \lambda q:(\sim\sim\underbrace{\sim\sim(A^\circ \vee B^\circ)}_{(A \vee B)^\circ}).\;
    \text{tne }q \\
    & \text{dne}_{\neg A} &&= \lambda q : \sim\sim\underbrace{\sim A^\circ}_{(\neg A)^\circ}.\; \text{tne }q
    \end{alignat*}

}
\section{Dependent Types}
{

    Types can include terms: merge the grammars, use judgements to decide if something is a type or a term.

    Some terms constitute proofs: \(\text{refl }e\) is a proof that \textit{something}?

    \begin{alignat*}{5}
    &\text{Terms } &&A,e &&::=\; &&x \mid \langle\rangle \mid 1 \mid \lambda x:A.\;e \mid \Pi x:A.\;B \mid e\;e' \mid \\
    &&&&&&&\text{refl }e \mid \text{subst}[x:A.\;B](e,e') \mid (e=e':A) && \\
    &\text{Contexts } &&\Gamma &&::=\; &&\cdot \mid \Gamma,x:A &&
    \end{alignat*}

    \begin{tabular}{l | l}
    Judgement & Description \\
    \hline
    \(\Gamma\vdash A\text{ type}\) & \(A\) is a type \\
    \(\Gamma\vdash e : A\) & \(e\) has type \(a\) \\
    \(\Gamma\vdash A \equiv B\text{ type}\) & \(A\) and \(B\) are identical types \\
    \(\Gamma\vdash e \equiv e' : A\) & \(e\) and \(e'\) are equal terms of type \(A\) \\
    \(\Gamma\text{ ok}\) & \(\Gamma\) is a well-formed context
    \end{tabular}

    \vspace{5mm}

    \begin{minipage}[t]{0.34\textwidth}
    %\footnotesize
    \centering
    Type Formation
    \begin{gather*}
    \prftree[r]{}
    {\Gamma\vdash 1\text{ type}}
    \\
    \\
    \prftree[r]{}
    {\Gamma\vdash A\text{ type}}
    {\Gamma,x:A\vdash B\text{ type}}
    {\Gamma\vdash\Pi x:A.\;B\text{ type}}
    \\
    \\
    \prftree[r]{}
    {\Gamma\vdash A\text{ type}}
    {\Gamma\vdash e:A}
    {\Gamma\vdash e;:A}
    {\Gamma\vdash (e=e':A)\text{ type}}
    \end{gather*}
    \end{minipage}
    \begin{minipage}[t]{0.65\textwidth}
    %\footnotesize
    \centering
    Rules
    \begin{gather*}
    \prftree[r]{}
    {\Gamma\vdash \langle\rangle : 1}
    \qquad
    \prftree[r]{}
    {x:A \in \Gamma}
    {\Gamma\vdash x : A}
    \\
    \\
    \prftree[r]{}
    {\Gamma\vdash A\text{ type}}
    {\Gamma,x:A\vdash e:B}
    {\Gamma\vdash (\lambda x:A.\;e) : (\Pi x:A.\;B)}
    \qquad
    \prftree[r]{}
    {\Gamma\vdash e: (\Pi x:A.\;B)}
    {\Gamma\vdash e':A}
    {\Gamma\vdash e\;e' : B[e'/x]}
    \\
    \\
    \prftree[r]{}
    {\Gamma\vdash e:A}
    {\Gamma\vdash \text{refl }e : (e = e : A)}
    \\
    \\
    {
        \small
        \prftree[r]{}
        {\Gamma\vdash A\text{ type}}
        {\Gamma,x:A \vdash B\text{ type}}
        {\Gamma\vdash e : (e_1=e_2 : A)}
        {\Gamma\vdash e' : B[e_1/x]}
        {\Gamma\vdash \text{subst}[x:A.\;B](e,e') : B[e_2/x]}
    }
    \\
    \\
    \prftree[r]{}
    {\Gamma\vdash e : A}
    {\Gamma\vdash A \equiv B\text{ type}}
    {\Gamma\vdash e : B}
    \end{gather*}
    \end{minipage}

    Also rules defining \(\equiv\) with normal equivalence relation rules, and congruence rules defining equality
    between terms. Gives power to the final rule given above, as it allows computation \textit{within} types.

}
\section{Languages}
{
    \subsubsection*{Simply Typed Lambda Calculus}
    {
        \begin{alignat*}{4}
        &\text{Types } &&X &&::= 1 \mid 0 \mid X \times Y \mid X + Y \mid X \rightarrow Y && \\
        &\text{Terms } &&e &&::= x \mid \langle\rangle \mid \langle e,e \rangle \mid \text{fst } e \mid \text{snd } e
        \mid \text{abort} \mid \text{L } e \mid \text{R } e \mid \text{case}(e,\text{L }x\rightarrow
        e,\text{R}y\rightarrow e) \mid \lambda x:X.\;e \mid e\;e && \\
        &\text{Contexts } &&\Gamma &&::= \cdot \mid \Gamma,x:X &&
        \end{alignat*}

        \begin{gather*}
        \prftree[r]{(\(1_\text{I}\))}
        {\Gamma\vdash\langle\rangle:1}
        \qquad
        \prftree[r]{(\(0_\text{E}\))}
        {\Gamma\vdash e : 0}
        {\Gamma\vdash\text{abort }e : X}
        \qquad
        \prftree[r]{(Hyp)}
        {x:X \in \Gamma}
        {\Gamma\vdash x : X}
        \\
        \\
        \prftree[r]{(\(\rightarrow_I\))}
        {\Gamma, x:X \vdash e : Y}
        {\Gamma\vdash (\lambda x:X.\;e) : X \rightarrow Y}
        \qquad
        \prftree[r]{(\(\rightarrow_E\))}
        {\Gamma\vdash e_1 : X \rightarrow Y}
        {\Gamma\vdash e_2 : X}
        {\Gamma\vdash e_1\;e_2 : Y}
        \\
        \\
        \prftree[r]{(\(\times_\text{I}\))}
        {\Gamma\vdash e_1 : X}
        {\Gamma\vdash e_2 : Y}
        {\Gamma\vdash\langle e_1,e_2 \rangle: X \times Y}
        \qquad
        \prftree[r]{(\(\times_\text{E1}\))}
        {\Gamma\vdash e : X \times Y}
        {\Gamma\vdash\text{fst }e : X}
        \qquad
        \prftree[r]{(\(\times_\text{E2}\))}
        {\Gamma\vdash e : X \times Y}
        {\Gamma\vdash\text{snd }e : Y}
        \\
        \\
        \prftree[r]{(\(+_\text{I1}\))}
        {\Gamma\vdash e : X}
        {\Gamma\vdash \text{L }e : X + Y}
        \qquad
        \prftree[r]{(\(+_\text{I2}\))}
        {\Gamma\vdash e : Y}
        {\Gamma\vdash \text{R }e : X + Y}
        \\
        \\
        \prftree[r]{(\(+_\text{E}\))}
        {\Gamma\vdash e : X + Y}
        {\Gamma,x:X\vdash e_1 : Z}
        {\Gamma,y:Y\vdash e_2 : Z}
        {\Gamma\vdash\text{case }(e, \text{L }x \rightarrow e_1,\text{R }y \rightarrow e_2) : Z}
        \end{gather*}

        The typed lambda calculi are total: all terms terminate as the types prevent the construction of a term like
        \(\Omega\).

        \subsection*{Intuitionistic Propositional Logic}
        {
            STLC is equivalent to IPL.

            \begin{gather*}
            \prftree[r]{}
            {P\in\Psi}
            {\Psi\vdash P\text{ true}}
            \qquad
            \prftree[r]{}
            {\Psi\vdash\bot\text{ true}}
            {\Psi\vdash R\text{ true}}
            \qquad
            \prftree[r]{}
            {\Psi\vdash \top\text{ true}}
            \\
            \\
            \prftree[r]{}
            {\Psi\vdash P\text{ true}}
            {\Psi\vdash P \vee Q\text{ true}}
            \qquad
            \prftree[r]{}
            {\Psi\vdash Q\text{ true}}
            {\Psi\vdash P \vee Q\text{ true}}
            \qquad
            \prftree[r]{}
            {\Psi\vdash P \vee Q\text{ true}}
            {\Psi,P\vdash R\text{ true}}
            {\Psi,Q\vdash R\text{ true}}
            {\Psi\vdash R\text{ true}}
            \\
            \\
            \prftree[r]{}
            {\Psi\vdash P \wedge Q\text{ true}}
            {\Psi\vdash P \text{ true}}
            \qquad
            \prftree[r]{}
            {\Psi\vdash P \wedge Q\text{ true}}
            {\Psi\vdash Q \text{ true}}
            \qquad
            \prftree[r]{}
            {\Psi\vdash P \text{ true}}
            {\Psi\vdash Q \text{ true}}
            {\Psi\vdash P \wedge Q\text{ true}}
            \\
            \\
            \prftree[r]{}
            {\Psi,P\vdash Q\text{ true}}
            {\Psi\vdash P \Longrightarrow Q\text{ true}}
            \qquad
            \prftree[r]{}
            {\Psi\vdash P \Longrightarrow Q\text{ true}}
            {\Psi\vdash P\text{ true}}
            {\Psi\vdash Q\text{ true}}
            \end{gather*}

            Negation is \textbf{derived}: \(\neg P \stackrel{\text{def}}{=}P\Longrightarrow\bot\). Refuting \(P\) means
            proving that \(P\Longrightarrow\bot\).

            Intuitionistic because there's no law of the excluded middle: \(\neg P\vee P\) does not hold, nor does
            \(\neg\neg P\Longrightarrow P\).

        }
    }
    \subsection*{Polymorphic Lambda Calculus/System F}
    {

        \begin{alignat*}{4}
        &\text{Types } &&A &&::= \alpha \mid A \rightarrow A \mid \forall\alpha.\;A && \\
        &\text{Terms } &&e &&::= x \mid \lambda x:X.\;e \mid e\;e \mid \Lambda\alpha.\;e \mid
        e\;\alpha && \\
        &\text{Type Contexts } &&\Theta &&::= \cdot \mid \Theta,\alpha && \\
        &\text{Term Variable Contexts } &&\Gamma &&::= \cdot \mid \Gamma,x:X &&
        \end{alignat*}

        \begin{gather*}
        \prftree[r]{}
        {\alpha\in\Theta}
        {\Theta\vdash \alpha \text{ type}}
        \qquad
        \prftree[r]{}
        {\Theta\vdash A\text{ type}}
        {\Theta\vdash B\text{ type}}
        {\Theta\vdash A \rightarrow B \text{ type}}
        \qquad
        \prftree[r]{}
        {\Theta,\alpha\vdash A \text{ type}}
        {\Theta\vdash \forall\alpha.\;A \text{ type}}
        \\
        \\
        \prftree[r]{}
        {\Theta\vdash \cdot \text{ ctx}}
        \qquad
        \prftree[r]{}
        {\Theta\vdash \Gamma \text{ ctx}}
        {\Theta\vdash X \text{ type}}
        {\Theta\vdash \Gamma,x:X \text{ ctx}}
        \end{gather*}

        Types can have free variables, so we need to ensure types are well-scoped (\(\Theta\vdash A\text{ type}\)). We
        then need judgements to ensure that type contexts are well-formed (\(\Theta\vdash\Gamma\text{ ctx}\)) as they
        contain types, which have a well-formedness condition.

        \begin{gather*}
        \prftree[r]{}
        {x:A \in \Gamma}
        {\Theta;\Gamma\vdash x:A}
        \\
        \\
        \prftree[r]{}
        {\Theta\vdash A\text{ type}}
        {\Theta;\Gamma,x:A\vdash e : B}
        {\Theta;\Gamma\vdash (\lambda x:A.e) : A \rightarrow B}
        \qquad
        \prftree[r]{}
        {\Theta;\Gamma\vdash e : A \rightarrow B}
        {\Theta;\Gamma\vdash e' : A}
        {\Theta;\Gamma\vdash e\;e' : B}
        \\
        \\
        \prftree[r]{}
        {\Theta,\alpha;\Gamma\vdash e : B}
        {\Theta;\Gamma\vdash \Lambda\alpha.\;e : \forall\alpha.\;B}
        \qquad
        \prftree[r]{}
        {\Theta\vdash A\text{ type}}
        {\Theta;\Gamma\vdash e : \forall\alpha.\;B}
        {\Theta;\Gamma\vdash a\;A : B[A/\alpha]}
        \end{gather*}

        Important that whenever we're dealing with a type or a context, we have an appropriate well-formedness condition
        around: all theorems using \(A\) should also include \(\Theta\vdash A\text{ type}\), etc.

        \subsection*{Second-Order Intuitionistic Logic}
        {
            PLC is equivalent to SIL: but we don't know SIL, so that's... great.
        }
    }
}
\section{Proofs}
{
    \subsubsection*{Progress: let bindings}
    {
        If \(\cdot \vdash e : \tau\) then either \(e\) is a value or \(\exists e'.\; e \leadsto e'\).

        By structural induction on \(\vdash\):

        \begin{adjustwidth}{5mm}{0mm}
            \dots
    
            Case \(\prftree[r]{(let)}{\Gamma\vdash e_1 : t'}{\Gamma,x:t'\vdash e_2 : t}{\Gamma\vdash \text{let }x =
            e_1 \text{ in } e_2 : t}\):

            \begin{adjustwidth}{5mm}{0mm}
                Assume \(\cdot\vdash e_1 : t'\) and \(x:t' \vdash e_2 : t\) from the top of the rule. Progress holds for
                them as they're subderivations.

                RTP Progress, so assume \(\cdot \vdash \text{let }x = e_1 \text{ in } e_2 : t\) and need to prove that
                the let-binding is either a value or that it steps.
                
                Using Progress on \(e_1\), either \(e_1\) is a value or \(\exists e_1'.\; e_1 \leadsto e_1'\).

                Case \(e_1\) is a value:
                \begin{adjustwidth}{5mm}{0mm}
                    By \(\prftree[r]{(op-let-1)}{\text{let }x = e_1 \text{ in } e_2 \leadsto e_2[e_1/x]}\), \(\text{let
                    }x = e_1 \text{ in } e_2 \leadsto e_2[e_1/x]\).
                \end{adjustwidth}

                Case \(\exists e_1'.\;e_1 \leadsto e_1'\):
                \begin{adjustwidth}{5mm}{0mm}
                    By \(\prftree[r]{(op-let-2)}{e_1 \leadsto e_1'}{\text{let }x = e_1 \text{ in } e_2 \leadsto
                    \text{let }x = e_1' \text{ in } e_2}\), \(\text{let }x = e_1 \text{ in } e_2 \leadsto \text{let }x =
                    e_1' \text{ in } e_2\).
                \end{adjustwidth}
            \end{adjustwidth}

            \dots
        \end{adjustwidth}
    }
}
\end{document}
