\documentclass[a4paper, 11pt]{article}

\usepackage{amsmath}
\usepackage{amssymb}
\usepackage{titlesec}
\usepackage[utf8]{inputenc}
\usepackage[margin=1.5cm]{geometry}
\usepackage{prftree}
\usepackage{changepage}
\usepackage{enumitem}
\usepackage{minted}
\usepackage{lmodern}
\usepackage{graphicx}
\usepackage{wrapfig}
\usepackage{ulem}
\usepackage{marvosym}
\usepackage{xcolor}
\usepackage{mathtools}
\usepackage{bm}

\title{\vspace{-2cm}Mobile Sensor Systems\vspace{-1.5cm}}
\author{}
\date{}

\setlength{\parindent}{0cm}
\setlength{\parskip}{2mm}
\setlist{nosep}

% Make ~ look more normal
\let\oldsim\sim
\renewcommand{\sim}{{\oldsim}}

\newmintinline[monospace]{text}{escapeinside=\#\#, mathescape, fontsize=\normalsize}
\newminted[monospacefigure]{text}{frame=lines, framesep=1mm, autogobble, escapeinside=\#\#, mathescape, breaklines}

\titlespacing{\section}{0mm}{2mm}{2mm}
\titlespacing{\subsection}{0mm}{2mm}{2mm}
\titlespacing{\subsubsection}{0mm}{2mm}{2mm}

\begin{document}
\maketitle

\section*{Background}
{
    12 billion mobile devices -- almost everyone has phones, now sensors and IOT etc.\ is gaining traction.

    The internet is only available to most people through phones, no computers.

    Mobile data traffic is \(\approx\)0.5 \textbf{exabytes} per day.

    In the US, \(>\)50\% of internet access is through WiFi, rest is mostly 4G. In Nigeria 75\% is through 3G. Geographic areas have vastly different access patterns based on the infrastructure available and economics.

    Phones have a multitude of sensors:
    \begin{itemize}
    \item Camera
    \item Microphone
    \item Fingerprint
    \item Accelerometer
    \item Gyroscope
    \item Heart Rate
    \end{itemize}

    Mobile devices are resource- and energy-constrained. Connectivity is highly variable in \textbf{performance} and \textbf{reliability}. Mobile devices are inherently \textbf{less secure} (wireless communication means broadcast, which is snoopable).

    Types of connectivity, with energy/reliability/rate limits:
    \begin{itemize}
    \item Cellular (SMS, 3G, 4G, ...)
    \item WiFi
    \item Bluetooth
    \item NFC
    \item Generic other radio communication
    \end{itemize}

    Wireless communication is generally organised as either:
    \begin{itemize}
    \item Infrastructure: Mobile devices connect to static base stations which are wired together into a trunk network. Handoff devices between base stations.
    \item Ad-hoc: No base stations, devices communicate between themselves when they're in range or available. Node organise themselves into a network.
    \end{itemize}

    Medium (radio spectrum) is generally multiplexed between multiple devices:
    \begin{itemize}
    \item Time: fixed or dynamic.
    \item Frequency: each communication gets a disjoint frequency band.
    \end{itemize}

    Multiplexing as above doesn't scale well with lots of devices or if communication is sparse. Ad-hoc approaches allow more statistical multiplexing:
    \begin{itemize}
    \item
    {
        CSMA/CD: transmit if nothing else is, jam if you detect a collision, random binary exponential backoff.
        \begin{itemize}
        \item Hidden Terminal:
        \end{itemize}
    }
    \end{itemize}
}
\end{document}